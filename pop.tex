\documentclass[9pt]{article}
\usepackage{inputenc}
\usepackage[hmargin=1.5cm,vmargin=2cm]{geometry} %set horizontal and vertical margins
\usepackage{setspace} %to change space between lines
\singlespace
% Math extension from the American Mathematical Society
\usepackage{amssymb} % high quality math symbols
\usepackage{amsmath} % high quality math formulas
\usepackage{booktabs} % publication quality tables
\usepackage{graphicx} % importation of graphics files
\usepackage{dcolumn}
\usepackage{lscape}
\usepackage[verbose]{placeins}
\usepackage[english]{babel}
\usepackage{blindtext}
\usepackage{graphicx,rotating,booktabs}
\usepackage{graphicx}
\usepackage{caption}
\usepackage{subcaption}
%Redefining sections as problems
\makeatletter
\newenvironment{exercise}{\@startsection
       {section}
       {1}
       {-.2em}
       {-3.5ex plus -1ex minus -.2ex}
       {2.3ex plus .2ex}
       {\pagebreak[3]%forces pagebreak when space is small; use \eject for better results
       \large\bf\noindent{Exercise }
       }
       }
       {%\vspace{1ex}\begin{center} \rule{0.3\linewidth}{.3pt}\end{center}}
       }
\makeatother
\renewcommand{\labelenumi}{(\alph{enumi})}
\renewcommand{\labelenumii}{\roman{enumii}}
\begin{document}

\title{Quantitative Political Analysis II, Homework 2}
\author{Alessandro Vecchiato}
\date{\today}
\maketitle

\begin{exercise}

\end{exercise}

\begin{exercise}
\begin{enumerate}
\item \begin{itemize}
\item The authors propose three assumptions. The first assumption is that, given that the CESS surveys are exhibit a short time period between the interviews and the intervening political change, it is plausible to assume that the {\it effects} of non-political background effects
\end{itemize}
\item 
\end{enumerate}
\end{exercise}
\end{document}